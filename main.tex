\documentclass[11pt]{article}

\usepackage{geometry, amsmath, amsthm, latexsym, amssymb, graphicx}
\geometry{margin=1in, headsep=0.25in}

\parindent 0in
\parskip 12pt

\begin{document}

\title{Cauchy-Schwarz Inequality}

\thispagestyle{empty}

\begin{center}
{\LARGE \bf Statistics}\\
Notes from classes in statistics courses at Uppsala University
Author: Emil Westin
\end{center}

\section{Time Series Analysis}


\textbf{Theorem (The Cauchy-Schwarz Inequality)}\\
Suppose $\vec{x},\vec{y} \in \mathbb{R}^n$. Then 
\begin{equation}
|\vec{x} \bullet \vec{y}| \le \|\vec{x}\| \|\vec{y}\| \label{cs-ineq}.
\end{equation}

\textbf{Proof}\\
First note that, if either $\vec{x}$ or $\vec{y}$ is the zero vector, 
then $\vec{x} \bullet \vec{y} = 0$ and $\|\vec{x}\| \|\vec{y}\| = 0$. 
In this case the theorem is trivially true because 
$|\vec{x} \bullet \vec{y}| = |0| = 0 = \|\vec{x}\| \|\vec{y}\|$.

Suppose, then, that neither $\vec{x}$ nor $\vec{y}$ is the zero vector. 
We will establish the truth of an inequality equivalent to \eqref{cs-ineq}, 
namely
\begin{equation}
-\|\vec{x}\| \|\vec{y}\| \le \vec{x} \bullet \vec{y} \le 
\|\vec{x}\| \|\vec{y}\|. \label{cs-ineq-alt}
\end{equation}


\clearpage

This is where you should begin Step 2.
\end{document}